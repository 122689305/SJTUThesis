%# -*- coding: utf-8-unix -*-
\chapter{搭建模板编译环境}

\section{安装TeX发行版}

\subsection{Mac OS X}

Mac用户可以从MacTeX主页\footnote{\url{https://tug.org/mactex/}}下载最新安装包。
也可以通过brew包管理器\footnote{\url{http://caskroom.io}}安装MacTeX。

\begin{lstlisting}[basicstyle=\small\ttfamily, numbers=none]
brew cask install mactex
\end{lstlisting}

\subsection{RedHat/CentOS}

建议RedHat/CentOS用户使用TeXLive主页\footnote{\url{https://www.tug.org/texlive/}}的脚本来安装TeXLive发行版。
以下命令将把TeXLive发行版安装到当前用户的家目录下。
若计划安装一个供系统上所有用户使用的TeXLive,请使用root账户操作。

\begin{lstlisting}[basicstyle=\small\ttfamily, numbers=none]
wget http://mirror.ctan.org/systems/texlive/tlnet/install-tl-unx.tar.gz
tar xzvpf install-tl-unx.tar.gz
cd install-tl-20150411/
./install-tl
\end{lstlisting}

\subsection{Deepin}

建议Deepin用户使用系统自带的包管理器安装TeXLive发行版:

\begin{lstlisting}[basicstyle=\small\ttfamily, numbers=none]
sudo apt-get install update
sudo apt-get install -y upgrade
sudo apt-get install -y texlive-full
\end{lstlisting}

\section{安装中文字体}

\subsection{Mac OS X}

Mac用户双击字体文件即可安装字体。

\subsection{Linux用户}

Linux用户请先将字体文件复制到字体目录下,调用fc-cache刷新缓存后即可在TeXLive中使用新字体。

\begin{lstlisting}[basicstyle=\small\ttfamily, numbers=none]
mkdir ~/.fonts
cp *.ttf ~/.fonts				# 当前用户可用新字体
cp *.ttf /usr/share/fonts/local/	# 所有用户可以使用新字体
fc-cache -f
\end{lstlisting}

