%%==========================
%% chapter01.tex for SJTU Master Thesis
%%==================================================

%\bibliographystyle{sjtu2} %[此处用于每章都生产参考文献]
\chapter{这是什么}
\label{chap:what}

这是上海交通大学(非官方)学位论文 \LaTeX 模板,当前版本是 \version 。
这个模板能够生成符合排版规范的学士、硕士和博士学位论文,并提供双面打印和单面打印的开关选项。
参考文献也使用了国标GBT7714风格。

\section{模板的来历}

最早的一版交大学位论文 \LaTeX 模板是一位热心的物理系同学制作的。
那份模板参考了自动化所学位论文模板,使用了CASthesis.cls文档类,中文字符处理则采用当时最为流行的 CJK\LaTeX 方案。
我根据交大研究生院对学位论文的要求进行了调整,完成了一份基本可用的交大 \LaTeX 学位论文模板。

但是,搭建一个 CJK\LaTeX 环境并不简单,在Linux下配置环境和调用中文字体的流程,对我而言犹如梦魇一般。
在William Wang的建议下,我开始着手把模板移植到 \XeTeX 上。
他完成了最初的移植,谢谢他的出色工作,使得后续的工作比我预想中的顺利。

这个学位论文模板从我用它来完成学位学位论文以后,就没有更新过。
过了快两年,随着知识水平的提高,我又想断断续续再做一些完善模板的工作,因此对原有的硕士论文模板做了修改,并在此基础上做了交大学士和博士学位论文 \LaTeX 模板。

现在,交大学位论文 \LaTeX 模板的代码在github
\footnote{\url{https://github.com/weijianwen/sjtu-thesis-template-latex}}
上维护。你可以\href{https://github.com/weijianwen/sjtu-thesis-template-latex/issues}{在github上开issue}
、或者在\href{https://bbs.sjtu.edu.cn/bbsdoc?board=TeX_LaTeX}{水源LaTeX版}发帖来反映遇到的问题。

\subsection{系统要求}
\label{sec:requirements}

模板在Windows,Linux和Mac OS X下测试通过,更详细的系统要求请参考\ref{sec:requirements}。

要使用这个模板协助你完成学位论文的创作,TeX系统中需要安装TeX Gyre Termes字体
\footnote{TeX Gyre Termes字体可以从\href{http://www.gust.org.pl/projects/e-foundry/tex-gyre/termes}{http://www.gust.org.pl/projects/e-foundry/tex-gyre/termes}下载。}
此外,还需要四款Adobe中文字体\footnote{请从合法渠道获得Adobe字体。}:AdobeSongStd、AdobeKaitiStd、AdobeHeitiStd、AdobeFangsongStd。
在Windows、Mac OS X 以及 Linux 上安装额外的字体,可以参考\href{https://www.searchfreefonts.com/articles/how-to-install-fonts.htm}{“How to install fonts?”}。

由于模板使用的中文解决方案是 \XeTeX/\LaTeX{},因此你使用的TeX发行版需要带有XeTeX支持。 
我在手头的几个 \TeX 环境上都做过测试,MacTeX 2011, TeXLive 2011和C\TeX 2.9(以及以后的版本)都能够顺利编译。
 
此外,这不是一份“从零开始的\LaTeX{}入门文档”。
在使用本模板前,你必须有一定的\LaTeX{}使用经验。

\subsection{模板文件}
\label{sec:layout}

\begin{lstlisting}[basicstyle=\small\ttfamily,caption={模板文件布局},label=layout,float,numbers=none]
├── HOWTO.pdf
├── Makefile
├── README.md
├── bib
│   ├── chap1.bib
│   └── chap2.bib
├── bst
│   └── GBT7714-2005NLang.bst
├── figure
│   └── sjtubanner.png
│   └── chap2
│       ├── testeps.eps
│       ├── testjpg.jpg
│       ├── testpdf.pdf
│       └── testpng.png
├── sjtuthesis.cfg
├── sjtuthesis.cls
├── statement.pdf
├── tags
├── tex
│   ├── abstract.tex
│   ├── ack.tex
│   ├── app_eq.tex
│   ├── app_log.tex
│   ├── chapter01.tex
│   ├── chapter02.tex
│   ├── chapter03.tex
│   ├── conclusion.tex
│   ├── id.tex
│   ├── projects.tex
│   ├── pub.tex
│   └── symbol.tex
└── thesis.tex
\end{lstlisting}

几个主要文件的功能简述如下。

\subsubsection{格式控制文件}
\label{sec:format}

格式控制文件控制着论文的表现形式,包括以下几个文件:
sjtuthesis.cfg, sjtuthesis.cls和GBT7714-2005NLang.bst。
其中,``.cfg''和``.cls''控制论文主体格式,``.bst''控制参考文献条目的格式,

\subsubsection{主控文件thesis.tex}
\label{sec:thesistex}

主控文件thesis.tex的作用就是将你分散在多个文件中的内容``整合''成一篇完整的论文。
使用这个模板撰写学位论文时,你的学位论文内容和素材会被``拆散''到各个文件中:
譬如各章正文、各个附录、各章参考文献等等。
在thesis.tex中通过``include''命令将论文的各个部分包含进来,从而形成一篇结构完成的论文。
封面页中的论文标题、作者等中英文信息,也是在thesis.tex中填写。
部分可能会频繁修改的设置,譬如行间距、图片文件目录等,我也放在了thesis.tex中。
你也可以在thesis.tex中按照自己的需要引入一些的宏包
\footnote{我对宏包的态度是:只有当你需要在文档中使用那个宏包时,才需要在导言区中用usepackage引入该宏包。如若不然,通过usepackage引入一大堆不被用到的宏包,必然是一场灾难。由于一开始没有一致的设计目标,\LaTeX 的各宏包几乎都是独立发展起来的,因重定义命令导致的宏包冲突屡见不鲜。}
。

大致而言,在thesis.tex中,大家只要留意把``章''一级的内容,以及各章参考文献内容包含进来就可以了。
需要注意,处理文档时所有的操作命令 \cndash{} xelatex, bibtex等,都是作用在thesis.tex上,而\emph{不是}后面这些``分散''的文件,请参考\ref{sec:process}小节。

\subsubsection{论文主体文件夹body}
\label{sec:thesisbody}

这一部分是论文的主体,是以``章''为单位划分的。

正文前部分(frontmatter):中英文摘要(abstract.tex)。其他部分,诸如中英文封面、授权信息等,都是根据thesis.tex所填的信息``画''好了,
不单独弄成文件。

正文部分(mainmatter):自然就是各章内容chapter\emph{xxx}.tex了。

正文后的部分(backmatter):附录(app\emph{xx}.tex);致谢(thuanks.tex);攻读学位论文期间发表的学术论文目录(pub.tex);个人简历(resume.tex)。
参考文献列表是``生成''的,也不作为一个单独的文件。另外,学校的硕士研究生学位论文模板中,也没有要求加入个人简历,所以我没有在thesis.tex中引入resume.tex。

\subsubsection{图片文件夹figure}
\label{sec:fig}

fig文件夹放置了需要插入文档中的图片文件(PNG/JPG/PDF/EPS),建议按章再划分子目录。

\subsubsection{参考文献数据库文件夹bib}
\label{sec:bib}

reference文件夹放置的是各章``可能''会被引用的参考文献文件。
参考文献的元数据,例如作者、文献名称、年限、出版地等,会以一定的格式记录在纯文本文件.bib中。
最终的参考文献列表是BibTeX处理.bib后得到的,名为thesis.bbl。
将参考文献按章划分的一个好处是,可以在各章后生成独立的参考文献,不过,现在看来没有这个必要。
关于参考文献的管理,可以进一步参考第\ref{chap:example}章中的例子。

\subsection{如何使用模板}
\label{sec:process}

模板的 \LaTeX 源文件需要用 \XeTeX 编译产生PDF文件。
我在此给出三种命令行下的编译方式:逐行手工执行、使用脚本、使用latexmk,大家可以根据自己的喜好,选择\emph{其中一种}完成工作。

\subsubsection{逐行手工执行}

模板使用 \XeTeX 引擎提供的xelatex的命令处理,作用于“主控文档”thesis.tex。并且,可以省略扩展名。
在命令提示符下逐行敲入如下命令完成编译。

\begin{lstlisting}[basicstyle=\small\ttfamily, caption={手动执行编译过程}, numbers=none]
xelatex -no-pdf --interaction=nonstopmode thesis
bibtex thesis 
xelatex -no-pdf --interaction=nonstopmode thesis
xelatex --interaction=nonstopmode thesis 
\end{lstlisting}

运行bibtex的时候会提示一些错误,猜测是{{\sc Bib}\TeX}对UTF-8支持不充分,一般不影响最终结果。留意因为拼写错误导致的``找不到文献错误''即可。

基本处理流程就是这样,一些 \LaTeX 排版的小例子可以参考第二章。

\subsubsection{使用GNU make编译}

模板自带了一个简单的Makefile,使用make可以方便地完成相应任务,如 pdf, view, clean, distclean等。

\begin{lstlisting}[basicstyle=\small\ttfamily, caption={使用GNU make编译}, numbers=none]
make clean all
\end{lstlisting}

对于CJK\LaTeX{}用户,附录\ref{chap:whydvipdfm}提供了简要的转换说明。

\section{对模板布局的说明}
\label{sec:thesisformat}

所有关于学位论文模板的要求,我参考下面的文献
\href{http://www.gs.sjtu.edu.cn/policy/fileShow.ahtml?id=130}{《上海交通大学研究生学位论文格式的统一要求 》}。
``要求''中的一些要求也不仅合理,譬如,公式和公式编号之前要用……连接,实现起来困难,看起来也不美观,从来没有人这样用,所以没有采纳。

