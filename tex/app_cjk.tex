\chapter{从{\CJKLaTeX}转向{\XeTeX}}
\label{chap:whydvipdfm}

我习惯把v0.2a使用dvipdfmx编译的硕士学位论文模板称为“{\CJKLaTeX}模板”,而这个使用{\XeTeX}引擎(xelatex程序)处理的模板则被称为“{\XeTeX/\LaTeX}模板”。
从{\CJKLaTeX}模板迁移到{\XeTeX\LaTeX}模板的好处有下:
\begin{enumerate}
\item[\large\smiley] 搭建{\XeTeX}环境比搭建{\CJKLaTeX}环境更容易;
\item[\large\smiley] 更简单的字体控制;
\item[\large\smiley] 完美支持PDF/EPS/PNG/JPG图片,不需要“bound box(.bb)”文件;
\item[\large\smiley] 支持OpenType字体的复杂字型变化功能;
\end{enumerate}

当然,这也是有代价的。由于{\XeTeX}比较新,在我看来,使用{\XeTeX}模板所必须付出的代价是:

\begin{enumerate}
\item[\large\frownie] 必须把你“古老的” \TeX 系统更新为较新的版本。TeXLive 2012和CTeX 2.9.2能够编译这份模板,而更早的版本则无能为力。
\item[\large\frownie] 需要花一些时间把你在老模板上的工作迁移到新模板上。
\end{enumerate}

第一条就看你如何取舍了,新系统通常意味着更好的兼容性,值得升级。而转换模板也不是什么特别困难的事情,可以这样完成:

\begin{enumerate}
\item 备份你要转换的源文件,以防你的工作成果丢失;
\item 将你原来的tex以及bib文件另存为UTF-8编码的文件。iconv、vim、emacs、UEdit等等工具都可以完成。WinEdt对文件编码识别功能很差(到了v6.0还是如此),不推荐作为字符编码转换工具;
\item 将diss.tex导言区中的内容替换为XeTeX模板diss.tex导言区的内容;
\item 将你对原先导言区的修改,小心翼翼地合并到新的导言区中;
\item 使用XeTeX模板中的GBT7714-2005NLang.bst替换原有的bst文件,新的bst文件只是将字符编码转换为UTF-8;
\item 删除bouding box文件;
\item 使用本文\ref{sec:process}介绍的方法,重新编译文档;
\end{enumerate}

